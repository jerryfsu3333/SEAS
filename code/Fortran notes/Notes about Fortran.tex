\documentclass[12pt]{article}
\usepackage{graphicx}
\usepackage{subcaption}
\usepackage{tocloft}
\usepackage{amssymb}
\usepackage{amsmath}
\usepackage[hmargin={1in, 1in}, vmargin={0.7in, 1in}]{geometry}
\usepackage{enumitem}
\usepackage[thinlines]{easytable}
\usepackage{bm}
\DeclareMathOperator*{\argmax}{argmax}
\DeclareMathOperator*{\argmin}{argmin}
%Input the title
\title{Some notes about the Fortran code in msda function}
\author{Jing Zeng}

\begin{document}
\maketitle
	


\section{Summary}

\begin{enumerate}[label={(\arabic*)}]
	\item The meaning of some parameters:
	\begin{itemize}
		\item \verb|obj|: Value of the objective function
		\[
			\sum_{k=2}^{K}\{\beta_k^T\Sigma\beta_k - \beta_k^T(\mu_k - \mu_1)\} + \lambda\sum_{k=2}^{K}||\bm{\beta_{\cdot k}}||_2
		\]
		. The difference ratio ($obj_{new} - obj_{old}$)/$obj_{new} < sml$ is a criterion of converage
		
		\item \verb|nk|: K-1
		\item \verb|sigma|: $\hat{\Sigma}$
		\item \verb|delta|: $(\hat{\mu}_2 - \hat{\mu}_1, \cdots, \hat{\mu}_K - \hat{\mu}_1)$
		\item \verb|pf|: a penalty vector used in code, is $\underbrace{(1,1,\cdots,1)}_{p}$ by default
		\item \verb|pmax|: During the process of iteration with each $\lambda$, we have to make sure the number of non-zero group variables $ni < pmax$, otherwise the process will stopped and return error code $-10000+l$ where $l$ is the order of current $\lambda$
		\item \verb|dfmax|: With each $\lambda$, if the converage is achieved and $ni < pmax$ thorough the whole process, we still need to check the number of non-zero variables, i.e. me, since some non-zero variables may degenerate to 0 again while $ni$ kept unchanged. If $me > dfmax$, which means the number of non-zero variables is too large, error code $-20000+l$ is returned. And by default $dfmax = nobs$, which indicates that typically the number of non-zero variables need to be less than the number of observations.
		
		\item \verb|nlam|: the number of $\lambda$ candidates.
		\item \verb|flmin|: used to give a factor which help generate the sequence of $\lambda$s
		\item \verb|ulam|: just the sequence of $\lambda$s. It will be used only if a sequence of self-defined $\lambda$s is given
		\item \verb|eps|, \verb|sml|: two converage thresholds
		\item \verb|maxit|: the maximal iterations
		\item \verb|verbose|: should we print the process or not
		\item \verb|nalam|: how many $\lambda$ candidates are used. Since with some $\lambda$, there will pop out an error code, so the subsequent $\lambda$s won't be used
		\item \verb|theta|: $(\Sigma^{-1}(\mu_2-\mu_1), \cdots, \Sigma^{-1}(\mu_K-\mu_1))^T \in \mathbb{R}^{(K-1)\times p}$. The order of columns are shuffled, we need to use function \verb|formatoutput| to reorder it.
		\item \verb|ni|: The number of all the appeared non-zero variables during the process of converage.
		\item \verb|me|: The number of all the non-zero variables in the end
		\item \verb|m|: the positions of all the appeared non-zero variables, can be used to reorder the \verb|theta|
		\item \verb|ntheta|: a set of $ni$ for each visited $\lambda$s
		\item \verb|alam|: a set of visited $\lambda$s
		\item \verb|npass|: iterations
		\item \verb|jerr|: the error code. 
			\begin{itemize}
				\item 0: success
				\item -10001: $ni>pmax$ during the process with the first $\lambda$ candidate,
				\item -20001: $me > dfmax$ with the first $\lambda$ candidate
				\item 10000: $maxval(pf) <= 0$
				\item $-l$: $npass > maxit$ during the process with the $l$th $\lambda$ candidate
			\end{itemize}

	\end{itemize}

\item How to tune the parameter $\lambda$?
	\begin{enumerate}
		\item $\lambda_1 = 9.9e30$ by default
		\item $alf = flmin**(1/(nlam-1))$ and calculate another quantity $al$ based on \verb|pf| and 
		\verb|delta|, then $\lambda_2 = al\times alf$
		\item $\lambda_{k+1} = \lambda_k \times alf, k=2,\cdots, nlam-1$
	\end{enumerate}

\item During the process of converage with each $\lambda$ candidate, or say the outer loop, the only exit of this loop is $ni > pmax$ or converage achieved

\item Outside the outer loop, if the new \verb|theta| and the old \verb|theta| are not very close, we can still use the criterion ($obj_{new} - obj_{old}$)/$obj_{new} < sml$ to give it the second chance. If this condition holds, then we still think it as converged.

\item Here are some outputs I got when error codes are thrown:
\begin{itemize}
	\item \verb|ni = 391, me = 320, pmax = 440, dfmax = 210, jerr = -20001|
	\item \verb|ni = 441, pmax = 440, jerr = -10001|
\end{itemize}

\end{enumerate}

\section{How to compile a Fortran code}

\begin{enumerate}[label={(\arabic*)}]
	\item In R:
	\begin{enumerate}[label={(\roman*)}]
		\item Run \verb|module load intel| and \verb|module load R| in bash
		\item Run command \verb|R CMD SHLIB fortran_file_name| in bash environment to compile Fortran file and it will generate a \verb|.so| file and a \verb|.o| file.
		\item Add \verb|dyn.load(so_file_name)| in Rscript file and with \verb|.Fortran| function the Rscript file will apply the Fortran file.
	\end{enumerate} 
	\item In bash:
	\begin{enumerate}[label={(\roman*)}]
		\item Run \verb|module load intel| and \verb|module load R| in bash
		\item Write a Fortran file \verb|.f|
		\item Run command \verb|gfortran fortran_file_name [-o output_file_name]| in bash environment to compile Fortran file and it will generate a \verb|.out| file.
		\item Run \verb|./output_file_name| and the output will printed out.
	\end{enumerate} 

\end{enumerate}

\textbf{Note}: 

\begin{enumerate}[label = {(\alph*)}]
	\item  We can add \verb|WRITE (*,*) VARIABLE_NAME| in Fortran file so the values of corresponding variable will be printed out.
	\item \verb|R CMD BATCH| will generate another output file \verb|.Rout|.
\end{enumerate}


\end{document}

